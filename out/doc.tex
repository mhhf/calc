% Article template for Mathematics Magazine
% Revised 7/2002  Thanks for Greg St. George
\documentclass[12pt]{article}
\usepackage{amssymb}
\usepackage[ngerman]{babel}
\usepackage[utf8]{inputenc}
\usepackage{prftree}
\usepackage{amsmath}
\usepackage{stmaryrd}
\usepackage{pf2}
\usepackage{graphicx}
\usepackage{tikz}
\usepackage{dot2texi}
\usepackage{tabularx, array}
\usetikzlibrary{shapes,arrows}
\renewcommand{\baselinestretch}{1.2}
%This is the command that spaces the manuscript for easy reading
%todo
\usepackage[colorinlistoftodos,prependcaption,textsize=tiny]{todonotes}
\usepackage{xargs}                      % Use more than one optional parameter in a new commands
\newcommandx{\QUESTION}[2][1=]{\todo[linecolor=none,backgroundcolor=blue!15,bordercolor=none,#1]{\textbf{QUESTION: }#2}}

\pflongnumbers

\newcolumntype{C}{>{\centering\arraybackslash$}X<{$}}
\newcommand{\parr}{\rotatebox[origin=c]{180}{$\&$}}

\begin{document}




\subsubsection*{RuleZer}

\begin{tabularx}{\linewidth}{CCC}
  \prftree[r]{$Id$}{}{  * :  ? A     \vdash  * :  ? A     }
&
  \prftree[r]{$Prem$}{}{ \Gamma  \vdash \Delta  }
&
  \prftree[r]{$Partial$}{}{ \Gamma  \vdash \Delta  }
\end{tabularx}


\subsubsection*{RuleCut}

\begin{tabularx}{\linewidth}{CCC}
  \prftree[r]{$Cut$}{  \Theta    ,  * : A     \vdash \Delta  }{ \Gamma  \vdash  * : A   }{  \Gamma    , \Theta    \vdash \Delta  }
\end{tabularx}


\subsubsection*{RuleStruct}

\begin{tabularx}{\linewidth}{CCC}
  \prftree[r]{$PL$}{ (  \Gamma_1    , \Delta_1    )  ,  \Gamma_2    , \Delta_2      \vdash \Theta  }{ (  \Gamma_1    , \Gamma_2    )  ,  \Delta_1    , \Delta_2      \vdash \Theta  }
&
  \prftree[r]{$PR$}{ \Theta  \vdash (  \Delta_1    , \Gamma_1    )  ,  \Delta_2    , \Gamma_2      }{ \Theta  \vdash (  \Delta_1    , \Delta_2    )  ,  \Gamma_1    , \Gamma_2      }
&
  \prftree[r]{$AR$}{ \Gamma  \vdash (  \Delta_1    , \Delta_2    )  , \Delta_3    }{ \Gamma  \vdash  \Delta_1    ,  \Delta_2    , \Delta_3      }
\\\\
  \prftree[r]{$AR$}{ \Gamma  \vdash  \Delta_1    ,  \Delta_2    , \Delta_3      }{ \Gamma  \vdash (  \Delta_1    , \Delta_2    )  , \Delta_3    }
&
  \prftree[r]{$AL$}{ (  \Delta_1    , \Delta_2    )  , \Delta_3    \vdash \Gamma  }{  \Delta_1    ,  \Delta_2    , \Delta_3      \vdash \Gamma  }
&
  \prftree[r]{$AL$}{  \Delta_1    ,  \Delta_2    , \Delta_3      \vdash \Gamma  }{ (  \Delta_1    , \Delta_2    )  , \Delta_3    \vdash \Gamma  }
\\\\
  \prftree[r]{$IL_L$}{    \cdot    , \Gamma    \vdash \Delta  }{ \Gamma  \vdash \Delta  }
&
  \prftree[r]{$IL_L$}{ \Gamma  \vdash \Delta  }{    \cdot    , \Gamma    \vdash \Delta  }
&
  \prftree[r]{$IL_R$}{  \Gamma    ,   \cdot    \vdash \Delta  }{ \Gamma  \vdash \Delta  }
\\\\
  \prftree[r]{$IL_R$}{ \Gamma  \vdash \Delta  }{  \Gamma    ,   \cdot    \vdash \Delta  }
&
  \prftree[r]{$IR_L$}{ \Gamma  \vdash    \cdot    , \Delta    }{ \Gamma  \vdash \Delta  }
&
  \prftree[r]{$IR_L$}{ \Gamma  \vdash \Delta  }{ \Gamma  \vdash    \cdot    , \Delta    }
\\\\
  \prftree[r]{$IR_R$}{ \Gamma  \vdash  \Delta    ,   \cdot    }{ \Gamma  \vdash \Delta  }
&
  \prftree[r]{$IR_R$}{ \Gamma  \vdash \Delta  }{ \Gamma  \vdash  \Delta    ,   \cdot    }
\end{tabularx}


\subsubsection*{RuleU}

\begin{tabularx}{\linewidth}{CCC}
  \prftree[r]{$\otimes_L$}{  \Gamma    ,   * : A     ,  * : B       \vdash  * : C   }{  \Gamma    ,  * :  A    \otimes B       \vdash  * : C   }
&
  \prftree[r]{$\multimap_R$}{  \Gamma    ,  * : A     \vdash  * : B   }{ \Gamma  \vdash  * :  A    \multimap B     }
\end{tabularx}


\subsubsection*{RuleBin}

\begin{tabularx}{\linewidth}{CCC}
  \prftree[r]{$\otimes_R$}{ \Delta  \vdash  * : B   }{ \Gamma  \vdash  * : A   }{  \Gamma    , \Delta    \vdash  * :  A    \otimes B     }
&
  \prftree[r]{$\multimap_L$}{  \Delta    ,  * : B     \vdash  * : C   }{ \Gamma  \vdash  * : A   }{  \Gamma    ,  \Delta    ,  * :  A    \multimap B         \vdash  * : C   }
\end{tabularx}


\subsubsection*{RuleNat}

\begin{tabularx}{\linewidth}{CCC}
  \prftree[r]{$nat_0$}{}{ \Gamma  \vdash    0     :   Nat      }
&
  \prftree[r]{$nat_s$}{  \Gamma    ,  n  :   Nat        \vdash     S n   :   Nat      }
\end{tabularx}



\end{document}
